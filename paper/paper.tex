\documentclass{llncs}
\usepackage{amsmath,amssymb,calc,ifthen}
\usepackage{float}
%\usepackage{cancel}
\usepackage[table,usenames,dvipsnames]{xcolor} % for coloured cells in tables
\usepackage{tikz}
% Allows us to click on links and references!
\usepackage{hyperref}
\usepackage{url}
\hypersetup{
colorlinks,
citecolor=black,
filecolor=black,
linkcolor=black,
urlcolor=black
}
% Nice package for plotting graphs
% See excellent guide:
% http://www.tug.org/TUGboat/tb31-1/tb97wright-pgfplots.pdf
\usetikzlibrary{plotmarks,shapes}
\usepackage{amsmath,graphicx}
\usepackage{epstopdf}
\usepackage{caption}
\usepackage{subcaption}
\usepackage{graphicx}
% highlight - useful for TODOs and similar
\usepackage{color}
\newcommand{\hilight}[1]{\colorbox{yellow}{#1}}
\newcommand\ci{\perp\!\!\!\perp} % perpendicular sign
\newcommand*\rfrac[2]{{}^{#1}\!/_{#2}} % diagonal fraction
\newcommand\SLASH{\char`\\}
\usepackage{listings}
% margin size
\usepackage{pdfpages}
\usepackage{enumitem} % for nested enumerate numbers 1 1.1 1.1.1
% \usepackage{breqn}
% \usepackage[linesnumbered]{algorithm2e}
% \usepackage{algorithmicx,algpseudocode}
% \usepackage{wrapfig} % for allowing text wrapped around the algorithm
% \newcommand\mycommfont[1]{\footnotesize\ttfamily\textcolor{blue}{#1}}
% \SetCommentSty{mycommfont}

% \usepackage{titlesec}
% \titlespacing*{\section}
% {0pt}{5.5ex plus 1ex minus .2ex}{4.3ex plus .2ex}


\DeclareMathOperator*{\argmin}{arg\,min}
\DeclareMathOperator*{\argmax}{arg\,max}

\begin{document}

\definecolor{blue3}{HTML}{86B7FC} % med blue
\definecolor{blue1}{HTML}{B5F1FF} % light blue
\definecolor{blue2}{HTML}{E0F9FF} % very light blue

\title{Disease Knowledge Transfer across Alzheimer's Variants}
%
\titlerunning{Disease Knowledge Transfer across Alzheimer's Variants}  % abbreviated title (for running head)
%                                     also used for the TOC unless
%                                     \toctitle is used
%

% * <mrazvan22@gmail.com> 2018-03-02T16:41:47.321Z:
% 
% Author list: Me, Marco (writing+model code), Pere (helped with validation), Alex (writing+tadpole), Neil (tadpole),  Arman, Keir (DTI data), Seb, Danny
% 
% ^ <mrazvan22@gmail.com> 2018-03-02T16:55:06.003Z.

\author{*******************************************}

\institute{***************************************}


\maketitle              % typeset the title of the contribution

\begin{abstract}
The availability of multimodal biomarker measurements in large scale clinical data is fundamental for the application of current disease progression models of neurodegenerative disorders. Due to the lack of data and to the low sample size, disease progression modeling of rare dementias is currently very challenging. To overcome this problem, we introduce Disease Knowledge Transfer (DKT), a technique for transferring biomarker's progression models between Alzheimer's disease (AD) variants. We assume that different types of dementia affect overlapping brain regions, and thus present shared biomarker characteristics that can be transferred across diseases. We then implement this paradigm as a joint-disease generative model of biomarker progressions, disentangling disease-specific from  disease-agnostic biomarker relationships. We demonstrate DKT on independent datasets of 1) multimodal typical AD data (tAD), with large sample size and several biomarkers available over time, and 2) Posterior Cortical Atrophy (PCA) data, for which only MRI scans are available. DKT is able to predict, in tested PCA subjects, plausible population-level biomarkers for structural and molecular imaging biomarkers, for which no data were available. DKT may be a useful tool to analyse and understand rare forms of dementia for which multimodal data is not available or is limited. Moreover, by leveraging data from multiple diseases, DKT also has the potential to provide more accurate disease staging compared to traditional disease progression models.


\keywords{Disease Progression Model, Transfer Learning, Manifold Learning, Alzheimer's Disease, Posterior Cortical Atrophy}
\end{abstract}

\section{Introduction}


% biomarkers in alzheimer's -> measuring the evolution helps staging in clinical trials
Several image-based biomarkers for Alzheimer's disease (AD) are currently used to track the progression of pathology: AV45 Positron Emission Tomography (PET) measuring amyloid-plaque aggregation, AV1451 PET measuring tau tangle aggregation, Flourodeoxyglucose (FDG) PET measuring brain hypometabolism, Magnetic Resonance Imaging (MRI) measuring structural integrity and Diffusion Tensor Imaging (DTI) measuring connectivity integrity. Measuring the exact evolution of these biomarkers over the disease progression is fundamental for better understanding of the pathology, as well as for improving stratification in clinical trials through patient staging.

% hypothetical model -> dimention 1: ordering across modalities -> dimention 2: spatial ordering
A hypothetical model of disease progression has been previously published by \cite{jack2010hypothetical}, which proposes that the first biomarkers to become abnormal are measures of amyloid beta aggregation, followed by tau abnormalities, hypometabolism, structural MRI-based measures and finally cognitive decline. While this hypothetical model proposes a inter-modality ordering of biomarkers in typical AD, it has also been observed that within the same modality, biomarker measurements have a spatial sequence of abnormality that correlates with Braak stages: hippocampal volumes and entorhinal measures become abnormal first, followed by other structures within the temporal, parietal and frontal lobes.

Several data-driven methods have been proposed in order to reconstruct group-wise long term biomarker progressions from collections of short term individual biomarker measurements (\cite{lorenzi2017disease,young2014data,donohue2014estimating}). These approaches mostly rely on the estimation of a latent time reparameterization associated with each individual, often encoded by a translation (time shift) of the subject's measurements over the temporal axis.  This time reparameterization can be used as a proxy of disease staging associated with an individual biomarker profile relatively to the global disease progression model. 

% Flow: multimodal data needed for disease progression -> disease progression on rare dementias is challenging -> cannot apply model learned from other disease because of different spatial patterns -> but studies show at least some partial spatial overlap -> therefore transfer learning possible due to presence of "some overlap" -> benefits of transfer learning -> existing transfer learning literature
The availability of multimodal collections of biomarker's measurements across clinical groups is fundamental to the application of such disease progression models. For this reason, the application of disease progression modeling to rare dementia type is currently very challenging, due to the associated problem of missing biomarkers and low sample size.  Moreover, an average model of disease progression estimated from sporadic AD cases may not generalize to specific disease variants. For example, in Posterior Cortical Atrophy (PCA), posterior regions such as the occipital lobe and superior parietal regions have been shown to be affected early. While the spatial patterns are distinct between various forms of dementia, recent studies \cite{Ossenkoppele2014atrophy} have showed at least partial spatial-overlap between different clinical phenotypes, while others have further suggested that PCA and typical AD subjects lie on a continuum of phenotypical variation \cite{crutch2012posterior}. The presence of overlap in pathology patterns across different dementias suggests that it should be theoretically possible to perform transfer learning across the diseases. There are two key benefits to performing transfer learning across dementia variants: 1. biomarker evolutions can be estimated for rare dementias for which there is not enough data (unimodal datasets, few subjects, cross-sectional data only) and 2. adding extra information from other datasets can help with staging and biomarker trajectory estimation. While recent studies \cite{hon2017towards} performed transfer learning from generic image datasets to Alzheimer's disease, we are not aware of any studies that have tried to transfer knowledge across different types of Alzheimer's variants, either for diagnosis classification or for estimating continuous biomarker trajectories.


We propose Disease Knowledge Transfer (DKT), a generative joint model that estimates continuous multimodal progressions for multiple dementias simultaneously and which inherently performs transfer learning between the modelled dementia phenotypes. This is achieved by disentangling \emph{disease-specific} from \emph{disease-agnostic} biomarker relationships. We fit the DKT model to two datasets simultaneously: (1) the TADPOLE Challenge dataset containing subjects from the ADNI study with MRI, FDG-PET, DTI, AV45 and AV1451 scans and (2) MRI scans from patients with Posterior Cortical Atrophy (PCA) from our local centre. We then used the fitted model to predict plausible non-MRI trajectories for PCA patients. We finally validated the DTI trajectories in PCA using a small test set of 20 DTI scans from the PCA patients and controls from our local centre.

% * <alexandra.young@ucl.ac.uk> 2018-03-02T17:54:07.562Z:
% 
% > unsupervised
% Is it unsupervised or does it know which disease/dataset is which, in which case it's semi-supervised?
% 
% ^ <mrazvan22@gmail.com> 2018-03-02T18:05:17.625Z:
% 
% Model is indeed supervised with respect to dataset/disease, but unsupervised with respect to diagnosis within same disease (e.g. Control, MCI and AD in ADNI). I removed the supervised-ness though in order not to create confusion. 
%
% ^.


\begin{figure}[h]
 \centering
 \includegraphics[width=0.8\textwidth,trim=0 0 0 80,clip]{figures/disease_knowledge_transfer.png}
 \caption{Outline of the proposed framework for joint modelling of multiple diseases. We assume that each disease can be modelled as the evolution of abstract dysfunctionality scores (Y-axis, top row), each one related to different brain regions. Each region-specific dysfunctionality score then further models (X-axis, bottom row) the progression of several modality-specific biomarkers within that same region. The biomarker correlations within the bottom units are assumed to be disease agnostic and shared across all diseases modelled. Disease knowledge transfer can then be achieved via the disease-agnostic units.}
 \label{fig:diagram}
\end{figure}

\section{Method}

Fig. \ref{fig:diagram} shows the overall diagram of our proposed framework for joint modelling of diseases. We assume that each disease can be modelled as the evolution of abstract dysfunctionality scores, each one related to different brain regions (top row). Each dysfunctionality score is then modelled as the progression of several biomarkers within that same region, but having different modalities (bottom row). Each group of biomarkers in the bottom row will be called a \emph{functional unit}, because the correlations between biomarkers are related though common "function" in a disease agnostic way, since they are related to the same underlying brain region. Biomarker groupings into functional units are defined a-priori. We model the correlations within each unit using a Gaussian-Process (GP) based disease progression model (DPM) \cite{lorenzi2017disease}, which allows us to reconstruct unit-specific \emph{dysfunction} progression manifold (X axis), which can be used for staging subjects. Finally, we use the same GP model to express the progression within each disease (Figure 1, top) in terms of the dysfunction scores estimated within each functional unit. More precisely, the X-axis dysfunction scores from the functional units become Y-axis measurements in the disease specific models.

The model has a simple mathematical formulation. We assume a set of given biomarkers measurements $Y = [y_{ijk} | (i,j,k) \in \Omega]$ for subject $i$ at visit $j$ in biomarker $k$, where $\Omega$ is defined as the set of available biomarker measurements, since subjects can have missing biomarkers in at various visits. We assume that each subject $i$ at each visit $j$ has an underlying disease stage $s_{ij} = \beta_i + m_{ij}$, where $m_{ij}$ represents the months since baseline visit for subject $i$ at visit $j$ and $\beta_i$ represents the subject-spefic time shift. We futher denote by $\theta_k$ the parameters used to represent the trajectory for biomarker $k \in K$ within its functional unit $\psi(k)$, where $\psi$: \{1, ..., K\} $ \rightarrow \Lambda$ is a function that maps each biomarker $k$ to a unique functional unit $l \in \Lambda$ and $\Lambda$ is the set of functional units. Moreover, we denote by $\lambda_d^l$ the parameters for the trajectory of the dysfunction score corresponding to functional unit $l \in \Lambda$ in the space of disease $d$. These definitions allow us to formulate the likelihood for a single measurement $y_{ijk}$ as follows:

\begin{equation}
 p(y_{ijk}|\theta_k, \lambda_{d_i}^{\psi(k)}, \beta_i) = \sum_{\beta_i^{\psi(k)}} p(y_{ijk}| \beta_i^{\psi(k)}, \theta_k) p(\beta_i^{\psi(k)}| \lambda_{d_i}^{\psi(k)}, \beta_i)
\end{equation}

where $\beta_i^{\psi(k)}$ is a latent variable denoting the dysfunctionality score of subject $i$ in functional unit $\psi(k)$, where biomarker $k$ was assigned. Moreover, $d_i \in \mathbb{D}$ represents the index of the disease space where subject $i$ belongs, where $\mathbb{D}$ is the set of all diseases modelled. For example, MCI and tAD subjects from ADNI as well as tAD subjects from our local cohort can all be assigned $d_i=1$, while PCA subjects from our local dataset can be assigned $d_i=2$. Healthy controls can be assigned to either disease space. Extending the above model to multiple subjects, visits and biomarkers, we get the final model likelihood:
\begin{equation}
 p(y_{.,.,.}|\theta, \lambda, \beta) = \\ \prod_{(i,j,k) \in \Omega} \sum_{\beta_i^{\psi(k)}} p(y_{ijk}| \beta_i^{\psi(k)}, \theta_k) p(\beta_i^{\psi(k)}| \lambda_{d_i}^{\psi(k)}, \beta_i)
\end{equation}
where $\theta = [\theta_1, ..., \theta_K]$, $\lambda = [\lambda_d^{l}|l \in \Lambda, d \in \mathbb{D}]$ and $\beta =[\beta_1, ..., \beta_N]$ represents the stacked parameters for the trajectories of biomarker, dysfunctionality scores in disease $d$ and subject-specific time shifts. Modelling of the likelihood of measurements $y_{ijk}$ within each functional unit is done using the non-parametric GP model by \cite{lorenzi2017disease}:

\begin{equation}
p(y_{ijk}| \beta_i^{\psi(k)}, \theta_k) = f(m_{ij} + \beta_i^{\psi(k)}) + \epsilon
\end{equation}
where $f \approx GP(0, \Sigma_G)$ models the population-level trajectory. The trajectory is also constrained to be monotonically increasing by enforcing a prohibit-likelihood on the derivative process $\frac{df(t)}{dt}$ \cite{lorenzi2017disease}. The same model is also applied to express the likelihood of $p(\beta_i^{\psi(k)}| \lambda_d^{\psi(k)}, \beta_i)$, which can give an estimate of the trajectories of dysfunctionality scores in the model specific to disease $d$. As already explained in \cite{lorenzi2017disease}, model fitting is done by Expectation-Propagation through conjugate gradient descent. In our hierarchical design, we first fit the models within each functional units independently, and then fit the disease specific models.

\subsection{Data acquisition and Preprocessing}

We used 87 controls and 76 PCA subjects from our local dataset, which only had MRI scans. From TADPOLE/ADNI, out of the subjects that contained at least one FDG PET, AV45, AV1451 or DTI scan, we randomly selected a subset of 230 subjects, in order not to bias the numbers towards the typical AD disease.  Volumetric measures for each subject have been obtained using the Freesurfer software in both ADNI and our local dataset. For FDG, AV45 and AV1451 PET, we used already extracted SUVR measures from ADNI. For DTI, we used fractional anisotropy (FA) measures from white-matter regions adjacent to each lobe. For every lobe, we averaged the biomarker values for regions of interest within each lobe, regressed out age, gender, TIV and dataset (ADNI vs our local dataset) and then normalised the final biomarker values to lie within the [0,1] range.

\section{Results}

% Fig1: biomarker traj. over dysfunction scores in one functional unit -> Fig2: dysfunction trajectories over disease stage in the two disease models -> Fig3: inferred biomarker trajectories "directly" over disease stage  in PCA
Fig. \ref{fig:pcaTadDisSpace}A shows the estimated biomarker trajectories within the \emph{occipital unit} plotted over the dysfunction scores, along with samples from the model posterior and aligned subject data. The X-axis shows the dysfunctionality scores within the occipital unit, which represent estimated time-shifts, in months, from an arbitrary reference X=0, while the Y-axis shows biomarker values normalised to [0,1] range. The model shows a good data fit, and we can observe most PCA subjects having abnormal occipital volumes, thus leading to high occipital dysfunctionality scores, in line with the current understanding of PCA as affecting posterior regions \cite{crutch2012posterior}. Fig. \ref{fig:pcaTadDisSpace} also shows the progression of dysfunctionality scores over the disease stage for (B) typical AD and (C) PCA. In typical AD, we see that temporal dysfunction becomes abnormal earliest, while PCA shows early parietal dysfunction, again in line with previous findings in the literature \cite{crutch2012posterior} (J.C. Baron, Neuroimage, 2001). In Fig. \ref{fig:PCAtrajByModality}, we plot the inferred biomarker trajectories for PCA directly across the disease progression. We do this for five different modalities: MRI Volumes, DTI, FDG, AV45 and AV1451. The results again recapitulate known patterns in PCA, where trajectories from posterior regions become abnormal first in MRI, DTI and AV1451, with the exception of FDG. This might be due to the normalisation of the FDG trajectories on the Y-axis to the [0,1] range, which results in artificially more scaling to trajectories that contain less disease signal. We also notice two non-monotonic trajectories in the AV45 plot, which are due to bad starting points within parameter optimisation combined with low disease signal within these biomarkers.

\newcommand{\expFld}{figures}

% data fitting within one unit
% \begin{figure}
% \centering
% \includegraphics[width=0.8\textwidth, trim=90 0 110 0, clip]{figures/unit1_allTraj_tad-drcTinyPen1_JMD.png} 
% \caption{Estimated biomarker trajectories in the Occipital Unit. Subject data from ADNI and our local cohort are also shown. The X-axis, defined as the occipital dysfunctionality score, represents the time-shifts (in months) of each subject. Red lines represent samples from the trajectory posterior. The Y-axis measures biomarker values (normalised).}
% \label{fig:occipUnit}
% \end{figure}

% PCA vs tAD disease space
% \vspace{-2em}
\begin{picture}(5,5)
\put(0,-70){\textbf{\Large{A}}}
\end{picture}
\begin{figure}[H]
\centering
\begin{subfigure}{\textwidth}
\centering
\includegraphics[width=0.8\textwidth, trim=90 0 110 0, clip]{figures/unit1_allTraj_tad-drcTinyPen1_JMD.png} 
% \caption{}
% \label{fig:occipUnit}
\end{subfigure}

\begin{picture}(5,5)
\put(0,20){\textbf{\Large{B}}}
\end{picture}
\begin{subfigure}{0.47\textwidth}
\centering
% typical AD\\
\includegraphics[width=0.8\textwidth, trim=0 0 0 20, clip]{figures/tAD_trajSameSpace_tad-drcTinyPen1_JMD.png} 
% \caption{C}
\end{subfigure}
\begin{picture}(5,5)
\put(0,20){\textbf{\Large{C}}}
\end{picture}
\begin{subfigure}{0.47\textwidth}
\centering
% PCA\\
\includegraphics[width=0.8\textwidth, trim=0 0 0 20, clip]{figures/PCA_trajSameSpace_tad-drcTinyPen1_JMD.png} 
% \caption{PCA}
\end{subfigure}
\caption{(A) Estimated biomarker trajectories in the Occipital Unit. Subject data from ADNI and our local cohort are also shown. The X-axis, defined as the occipital dysfunctionality score, represents the time-shifts (in months) of each subject. Red lines represent samples from the trajectory posterior. The Y-axis measures biomarker values (normalised). (B-C) Progression of dysfunctionality scores for (B) typical AD and (C) PCA.}
\label{fig:pcaTadDisSpace}
\end{figure}



% estimated (hypothetical) trajectories in PCA: DTI, FDG, AV45, AV1451.Volumetric trajectories were based on PCA MRI data.
\begin{figure}
 \includegraphics[width=\textwidth, trim=0 20 0 0, clip]{figures/trajDisSpaceOverlap_PCA_tad-drcTinyPen1_JMD.png}
 \caption{Estimated trajectories for the PCA cohort. The only data that were available were the MRI volumetric data. The dynamics of the other biomarkers has been inferred by the model using data from typical AD, and taking into account the different spatial distribution of pathology in PCA as compared to typical AD.}
 \label{fig:PCAtrajByModality}
\end{figure}


\section{Validation}

We performed validation using a separate test set of 20 DTI scans from controls and PCA patients from our local cohort. Fig. \ref{fig:DTIvalid} shows the estimated DTI biomarker trajectories plotted directly against the disease progression time-shifts, along with the DTI data from the test set. The time shift of each subject was estimated only based on their MRI scans. The inferred trajectories in PCA show good agreement with the data in the cingulate, hippocampus and temporal lobes, but less agreement in the other regions. This might be because the dataset that was used to train the model had better disease signal in DTI regions more traditionally associated with typical AD.

% DTI validation with local dataset

\begin{figure}
 \includegraphics[width=\textwidth, trim=0 0 0 0, clip]{figures/validDtiPCA.png}
 \caption{Validation of the Disease Knowledge Transfer Model using DTI data of PCA subjects from our local cohort. Each subject has been shifted along the X-axis only according to their MRI data. No DTI data from PCA subjects was used to fit the model.}
\label{fig:DTIvalid}
\end{figure}

\section{Discussion}

% summary: what we developed -> what the model can do
We present here, for the first time, a framework for joint modelling of biomarker progression in multiple dementias simultaneously. The framework allows transferring biomarker trajectories to rare dementias for which there is not enough data to allow estimation of such trajectories, accounting for a different spatial distribution of pathology between dementia varieties. 

% limitations: each region needs to have disease signal -> validation did not work well in some regions -> not accounted for subject specific effects. 
Our work has several limitations: (1) the model assumes that within each region there is enough disease signal to properly estimate the dysfunctionality scores (2) in the validation, the inferred trajectories in posterior regions traditionally associated with PCA do not match well with the data and (3) in terms of modelling, our framework did not account for subject-specific effects.

% how to address aforementioned limitations + future work
There are several potential avenues for further research. During training, we can account for missing disease signal in some brain regions by adding other types of dementias that affect other regions. The estimated PCA trajectories also need to be thoroughly validated using data from other modalities. The model formulation can also be naturally extended to include subject-specific effects.

%
% ---- Bibliography ---- 
% USE HARVARD STANDARD

\bibliographystyle{unsrtnat}
\begin{thebibliography}{5}

\bibitem{jack2010hypothetical}
Jack, C.R., Knopman, D.S., Jagust, W.J., Shaw, L.M., Aisen, P.S., Weiner, M.W., Petersen, R.C. and Trojanowski, J.Q., 2010. Hypothetical model of dynamic biomarkers of the Alzheimer's pathological cascade. The Lancet Neurology, 9(1), pp.119-128.

\bibitem{lorenzi2017disease}
Lorenzi, M., Filippone, M., Alexander, D.C. and Ourselin, S., 2017. Disease Progression Modeling and Prediction through Random Effect Gaussian Processes and Time Transformation. arXiv preprint arXiv:1701.01668.

\bibitem{young2014data}
Young, A.L., Oxtoby, N.P., Daga, P., Cash, D.M., Fox, N.C., Ourselin, S., Schott, J.M. and Alexander, D.C., 2014. A data-driven model of biomarker changes in sporadic Alzheimer's disease. Brain, 137(9), pp.2564-2577.

\bibitem{donohue2014estimating}
Donohue, M.C., Jacqmin-Gadda, H., Le Goff, M., Thomas, R.G., Raman, R., Gamst, A.C., Beckett, L.A., Jack, C.R., Weiner, M.W., Dartigues, J.F. and Aisen, P.S., 2014. Estimating long-term multivariate progression from short-term data. Alzheimer's \& Dementia, 10(5), pp.S400-S410.

\bibitem{Ossenkoppele2014atrophy}
Ossenkoppele, R., Cohn‐Sheehy, B.I., La Joie, R., Vogel, J.W., Möller, C., Lehmann, M., van Berckel, B.N., Seeley, W.W., Pijnenburg, Y.A., Gorno‐Tempini, M.L. and Kramer, J.H., 2015. Atrophy patterns in early clinical stages across distinct phenotypes of Alzheimer's disease. Human brain mapping, 36(11), pp.4421-4437.

\bibitem{crutch2012posterior}
Crutch, S.J., Lehmann, M., Schott, J.M., Rabinovici, G.D., Rossor, M.N. and Fox, N.C., 2012. Posterior cortical atrophy. The Lancet Neurology, 11(2), pp.170-178.

\bibitem{hon2017towards}
Hon, M. and Khan, N., 2017. Towards Alzheimer's Disease Classification through Transfer Learning. arXiv preprint arXiv:1711.11117.


\end{thebibliography}

\clearpage

\end{document}

